% Options for packages loaded elsewhere
\PassOptionsToPackage{unicode}{hyperref}
\PassOptionsToPackage{hyphens}{url}
%
\documentclass[UTF8]{ctexart}
\usepackage{amsmath,amssymb}
\usepackage{iftex}
\ifPDFTeX
  \usepackage[T1]{fontenc}
  \usepackage[utf8]{inputenc}
  \usepackage{textcomp} % provide euro and other symbols
\else % if luatex or xetex
  \usepackage{unicode-math} % this also loads fontspec
  \defaultfontfeatures{Scale=MatchLowercase}
  \defaultfontfeatures[\rmfamily]{Ligatures=TeX,Scale=1}
\fi
\usepackage{lmodern}
\ifPDFTeX\else
  % xetex/luatex font selection
\fi
% Use upquote if available, for straight quotes in verbatim environments
\IfFileExists{upquote.sty}{\usepackage{upquote}}{}
\IfFileExists{microtype.sty}{% use microtype if available
  \usepackage[]{microtype}
  \UseMicrotypeSet[protrusion]{basicmath} % disable protrusion for tt fonts
}{}
\makeatletter
\@ifundefined{KOMAClassName}{% if non-KOMA class
  \IfFileExists{parskip.sty}{%
    \usepackage{parskip}
  }{% else
    \setlength{\parindent}{0pt}
    \setlength{\parskip}{6pt plus 2pt minus 1pt}}
}{% if KOMA class
  \KOMAoptions{parskip=half}}
\makeatother
\usepackage{xcolor}
\setlength{\emergencystretch}{3em} % prevent overfull lines
\providecommand{\tightlist}{%
  \setlength{\itemsep}{0pt}\setlength{\parskip}{0pt}}
\setcounter{secnumdepth}{-\maxdimen} % remove section numbering
\ifLuaTeX
  \usepackage{selnolig}  % disable illegal ligatures
\fi
\IfFileExists{bookmark.sty}{\usepackage{bookmark}}{\usepackage{hyperref}}
\IfFileExists{xurl.sty}{\usepackage{xurl}}{} % add URL line breaks if available
\urlstyle{same}
\hypersetup{
  pdftitle={3. 增量型本构方程的建立},
  hidelinks,
  pdfcreator={LaTeX via pandoc}}

\title{3. 增量型本构方程的建立}
\author{}
\date{}

\begin{document}
\maketitle

\hypertarget{ux4e00ux56deux987e-druckerux516cux8bbeux5efaux7acbux7684ux589eux91cfux516cux5f0f}{%
\subsubsection{一、回顾:
Drucker公设建立的增量公式}\label{ux4e00ux56deux987e-druckerux516cux8bbeux5efaux7acbux7684ux589eux91cfux516cux5f0f}}

{\[d\varepsilon_{ij} = \frac{1}{2\mu}dS_{ij} + \frac{1 - 2\nu}{E}d\sigma_{ij}\delta_{ij} + d\lambda\frac{\partial f}{\partial\sigma_{ij}}\]}其中:\\
{\[d\varepsilon_{ij}^{p} = d\lambda\frac{\partial f}{\partial\sigma_{ij}}\]}

\hypertarget{ux4e8cux589eux91cfux578bux672cux6784ux5173ux7cfbux5efaux7acbux5851ux6027ux6d41ux52a8ux7406ux8bba}{%
\subsubsection{二、增量型本构关系建立(塑性流动理论)}\label{ux4e8cux589eux91cfux578bux672cux6784ux5173ux7cfbux5efaux7acbux5851ux6027ux6d41ux52a8ux7406ux8bba}}

\hypertarget{ux76f8ux5173ux63a8ux5bfcux548cux6982ux5ff5ux8865ux5145}{%
\paragraph{(1)
相关推导和概念补充}\label{ux76f8ux5173ux63a8ux5bfcux548cux6982ux5ff5ux8865ux5145}}

(P-R)本构关系和(L-M)本构关系推导\\
强化材料的本构方程推导

Shield 和 Ziegler 指出,塑性本构关系的构成可以包含以下的方面:

\begin{enumerate}
\tightlist
\item
  初始屈服条件分析
\item
  加载函数(强化条件)
\item
  与初始屈服面和后继屈服面相关的流动法则
\end{enumerate}

塑性流动理论包括全量理论和增量理论

\hypertarget{ux5168ux91cfux7406ux8bbaux548cux589eux91cfux7406ux8bba}{%
\subparagraph{全量理论和增量理论}\label{ux5168ux91cfux7406ux8bbaux548cux589eux91cfux7406ux8bba}}

(1) 全量理论也称塑性形变理论,认为塑性状态下仍然存在应力应变的全量关系\\
主要是Hencky全量理论 -\textgreater{}
Ilyushin全量本构关系(考虑了弹性变形和强化关系)

(2)
增量理论也称塑性流动理论,主要研究塑性状态下应变率(增量)和应力率(增量)之间的关系,主要有:

\begin{enumerate}
\tightlist
\item
  Levy - Mises 理论 -\textgreater{} 适用于刚塑性变形
\item
  Prandtl - Reuss 理论 -\textgreater{} 考虑了弹性变形
\end{enumerate}

\hypertarget{prandtl-reussux548clevy-misesux6d41ux52a8ux6cd5ux5219}{%
\paragraph{(2)
Prandtl-Reuss和Levy-Mises流动法则}\label{prandtl-reussux548clevy-misesux6d41ux52a8ux6cd5ux5219}}

若使用Mises屈服条件,建立应力方程为\\
{\[\boxed{f(\sigma_{ij}) = \sigma_{i}^{2} - \sigma_{s}^{2} = 0}\]}代入Drucker公设建立的塑性应变项即得到\\
{\[\begin{matrix}
\boxed{d\varepsilon_{ij}^{p} = d\lambda \cdot S_{ij}} \\
\end{matrix}\]}

Caution

由上式以及{\(d\varepsilon_{ij}^{p} = \sqrt{\frac{3}{2}}\varepsilon_{i}^{p},S_{ij} = \sqrt{\frac{2}{3}}\sigma_{i}\)}得到本构关系

{\[d\lambda = \frac{3d\varepsilon_{i}^{p}}{2\sigma_{i}}\]}

注意这个本构关系适用于所有模型(包括线性强化,幂强化,等等)

并且可以在各个方向上列相应的方程并代入{\(d\lambda\)}

上式称为\textbf{Prandtl-Reuss流动法则}, 与Levy-Mises流动法则及其类似

Levy-Mises流动法则初始形式以及Levy-Mises本构关系

由于Levy-Mises流动法则是针对于理想刚塑性材料的,不参与 Drucker 公设,

因此弹性变形可以忽略去,即\\
{\[\varepsilon_{ij}^{p} = d\varepsilon_{ij}^{p} = d\lambda \cdot S_{ij}\]}

由上式容易推导得出应力强度{\(\sigma_{i}\)}和塑性应变强度{\(d\varepsilon_{ij}^{p}\)}之间的关系为\\
{\[d\lambda = \frac{3d\varepsilon_{i}^{p}}{2\sigma_{i}}\]}则得到\textbf{Levy-Mises本构关系:}\\
{\[\begin{matrix}
\boxed{d\varepsilon_{ij} = \frac{3d\varepsilon_{i}}{2\sigma_{i}}S_{ij}} \\
\end{matrix}\]}上式只适用于理想刚塑性材料

\hypertarget{ux589eux91cfux672cux6784ux5173ux7cfbplandtl-reussux672cux6784ux5173ux7cfb}{%
\paragraph{(3)
增量本构关系(Plandtl-Reuss本构关系)}\label{ux589eux91cfux672cux6784ux5173ux7cfbplandtl-reussux672cux6784ux5173ux7cfb}}

有{\(d\lambda\)}的表达式\\
{\[\boxed{d\lambda = \frac{3dW^{p}}{2\sigma_{s}^{2}}}\]}其中

\[\left\{ \begin{matrix}
{dW^{p} = \sigma_{ij}d\varepsilon_{ij}^{p} = \sigma_{i}d\varepsilon_{i}^{p}} \\
{dW^{d} = S_{ij}de_{ij}} \\
{dW^{p} = dW^{d}} \\
\end{matrix} \right.\]

故有本构关系\\
{\[\boxed{d\sigma_{ij} = \frac{1}{2\mu}dS_{ij} + \frac{1 - 2\nu}{E}d\sigma_{m}\delta_{ij} + d\lambda \cdot S}\]}或

\[\boxed{\left\{ \begin{matrix}
{de_{ij} = \frac{1}{2\mu}dS_{ij} + d\lambda \cdot S} \\
{d\varepsilon_{ii} = \frac{1 - 2\nu}{E}d\sigma_{ii}} \\
\end{matrix} \right.}\]

此两式称为\textbf{Prandtl- Reuss本构关系}

\hypertarget{ux7406ux60f3ux5f39ux5851ux6027ux672cux6784ux65b9ux7a0b}{%
\subparagraph{1.
理想弹塑性本构方程}\label{ux7406ux60f3ux5f39ux5851ux6027ux672cux6784ux65b9ux7a0b}}

{\(\lambda\)}的表达

\begin{itemize}
\tightlist
\item
  塑性应变表达\\
  {\[d\lambda = \frac{3}{2}\frac{d\varepsilon_{i}^{p}}{\sigma_{s}}\]}
\item
  塑性功表达\\
  {\[\qquad W^{p} = W^{d},\quad W^{p} = \sigma_{ij}\varepsilon_{ij}^{p} = \sigma_{i}\varepsilon_{i}^{p},\quad dW_{d} = S_{ij}de_{ij}\]}方程直接代入即可得到相应的增量本构关系\\
  {\[d\sigma_{ij} = \frac{1}{2\mu}dS_{ij} + \frac{1 - 2\nu}{E}d\sigma_{m}\delta_{ij} + \frac{3}{2}\frac{d\varepsilon_{i}^{p}}{\sigma_{s}} \cdot S\]}
\end{itemize}

\hypertarget{ux7406ux60f3ux521aux5851ux6027ux672cux6784ux65b9ux7a0b}{%
\subparagraph{2.
理想刚塑性本构方程}\label{ux7406ux60f3ux521aux5851ux6027ux672cux6784ux65b9ux7a0b}}

{\[d\lambda = \frac{3}{2}\frac{d\varepsilon_{i}}{\sigma_{s}}\]}增量型本构关系:\\
{\[d\varepsilon_{ij} = \frac{3d\varepsilon_{i}}{2\sigma_{s}}S_{ij}\]}

\hypertarget{ux5f3aux5316ux6750ux6599ux7684ux672cux6784ux65b9ux7a0b}{%
\subparagraph{3.
强化材料的本构方程}\label{ux5f3aux5316ux6750ux6599ux7684ux672cux6784ux65b9ux7a0b}}

{\[d\lambda = \frac{3d\sigma_{i}}{2H^{\prime}\sigma_{i}}\]}\textbf{其中{\(H^{\prime}\)}是{\(\sigma_{i} - \int\varepsilon_{i}^{p}\)}的图像直线斜率}\\
即:\\
{\[\boxed{d\sigma_{ij} = \frac{1}{2\mu}dS_{ij} + \frac{1 - 2\nu}{E}d\sigma_{m}\delta_{ij} + \frac{3d\sigma_{i}}{2H^{\prime}\sigma_{i}} \cdot S}\]}或者

\[\left\{ \begin{matrix}
{de_{ij} = \frac{1}{2\mu}dS_{ij} + \frac{3d\sigma_{i}}{2H^{\prime}\sigma_{i}} \cdot S} \\
{d\varepsilon_{ii} = \frac{1 - 2\nu}{E}d\sigma_{ii}} \\
\end{matrix} \right.\]

\end{document}
